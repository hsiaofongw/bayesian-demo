\documentclass{ctexart}

\usepackage{amsthm}

\newcommand{\pr}{\mathrm{Pr}}
\newtheorem{theorem}{定理}

\title{使用贝叶斯方法进行推断}
\begin{document}
    \maketitle

    \section{独立性和条件独立性}

    \subsection{独立性}

    对于事件$A$和事件$B$,我们说事件$A$和事件$B$是互相独立的,如果:
    \begin{equation}
        \pr(A \cap B) = \pr(A) \pr(B).
    \end{equation}
    \subsection{条件概率}
    当已知事件$A$发生时,事件$B$发生的概率,称为事件$B$关于事件$A$的条件概率,记做:
    \begin{equation}
        \pr(B|A).
    \end{equation}
    容易证明,对任意事件$A$、事件$B$,恒有:
    \begin{equation}
        \pr(A)\pr(B|A) = \pr(A \cap B)
    \end{equation}
    如此一来,我们得到了一种计算条件概率的方法:
    \begin{equation}
        \pr(B|A) = \frac{\pr(A \cap B)}{\pr(A)} = \frac{\pr(A|B)\pr(B)}{\pr(A)}
        \label{eq:condition-1}
    \end{equation}
    式中,我们要求$\pr(A) \neq 0$.

    事件的独立性可以用条件概率来表述,具体地,我们有
    \begin{theorem}
        设$A,B$是事件,则
        \begin{equation}
        \pr(A \cap B) = \pr(A)\pr(B) \iff \pr(A) = \pr(A|B).
        \end{equation}
    \end{theorem}
    \begin{proof}
        充分性.设给定$\pr(A) = \pr(A|B)$. 那么
        \begin{equation}
            \pr(A)\pr(B) = \pr(A|B)\pr(B) = \pr(A\cap B)
        \end{equation}
        从而充分性成立.

        必要性.设给定$\pr(A\cap B) = \pr(A) \pr(B)$. 那么,假设$\pr(B) \neq 0$,则
        \begin{equation}
            \frac{\pr(A \cap B)}{\pr(B)} = \frac{\pr(A)\pr(B)}{\pr(B)} = \pr(A)
        \end{equation}
        依式\ref{eq:condition-1},我们得到
        \begin{equation}
            \pr(A|B) = \pr(A)
        \end{equation}
        从而必要性成立.
    \end{proof}

    \subsection{条件独立性}
    条件独立性是说,当某个事件发生时,另外两个(或多个)事件互相独立.具体地,设$A,B,C$是事件,则事件$A$,事件$B$关于事件$C$条件独立是指:
    \begin{equation}
        \pr(A \cap B | C) = \pr(A | C) \pr(B | C)
    \end{equation}
    例如:「在吸烟的人群中,男性患肺癌的风险和女性患肺癌的风险相当」就是一个条件独立性的陈述.对于一个受访者,用$A$表示受访者是男性,$\neg A$表示受访者是女性,用$B$表示受访者患肺癌,$\neg B$表示受访者不患肺癌,用$C$表示吸烟,则上述论断可写为:
    \begin{equation}
        \pr(A \cap B | C) = \pr(A | C) \pr(B | C)
    \end{equation}
    我们亦有条件独立性的等价表述:
    \begin{theorem}
        设$A,B,C$是事件,则
        \begin{equation}
            \pr(A \cap B | C) = \pr(A | C) \pr(B | C) \iff \pr(A | B \cap C) = \pr(A | C)
        \end{equation}
    \end{theorem}
    \begin{proof}
        充分性.设给定$\pr(A|B\cap C)=\pr(A|C)$.注意到
        \begin{equation}
            \pr(A | B \cap C) = \frac{\pr(A \cap B \cap C)}{\pr(B \cap C)} = \frac{\pr(A\cap B \cap C)}{\pr(B|C) \pr(C)}
        \end{equation}
        所以
        \begin{equation}
            \frac{\pr(A\cap B \cap C)}{\pr(B|C)\pr(C)} = \pr(A|C)
        \end{equation}
        所以
        \begin{equation}
            \frac{\pr(A \cap B \cap C)}{\pr(C)} = \pr(A|C) \pr(B|C)
        \end{equation}
        对上列等式左边应用式\ref{eq:condition-1},得
        \begin{equation}
            \pr(A \cap B | C) = \pr(A|C)\pr(B|C)
        \end{equation}
        于是充分性得证.从上述证明过程倒推回去可得到必要性.
    \end{proof}
\end{document}